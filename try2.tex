\documentclass[UTF8]{ctexart}%中文短文所以使用ctexart,用[UTF8]选项说明编码
\title{\heiti 杂谈勾股定理}
\author{\kaishu 刘峻嘉}
%标题和作者的字体可以直接修改,即加\kaishu或\heiti等
\date{\today}


\bibliographystyle{plain}%声明参考文献的格式(见3.3.1节)
\usepackage{graphicx}    %插图功能由graphicx宏包提供,引入宏包后就可以用\includegraphics[]{}来命令插图了
\usepackage{float}
\usepackage{geometry}
\geometry{a4paper,centering,scale=0.8}  %geometry宏包用于设计页面尺寸
\usepackage[format=hang,font=small,textfont=it]{caption}    %用caption宏包可以用改变图表标题格式
\usepackage[nottoc]{tocbibind}          %用tocbibind宏包可以增加目录的项目,默认会在目录中加入目录项本身、参考文献、索引等项目

%begin前面的为导言区,用来对文档的性质做一些设置,或自定义一些命令

\begin{document}

\maketitle  %\maketitle实际输出论文标题
\begin{abstract}
     这是一篇关于勾股定理的小论文
\end{abstract}
\tableofcontents %命令输出目录
\section{古代的勾股定理}
起源于毕达哥拉斯,中国也早有研究过,见于欧几里德\footnote{欧几里德,约公元前330--275年。}《几何原本》
的第XX页%上行的\footnote{}代表脚注
······的数称为\emph{勾股数}。%\emph{}表示改变字体表强调
\section{勾股定理的现代形式}
\bibliography{math}  %提示TEX从文献数据库math中获取文献信息,打印参考文献列表
近代的形式比较统一了
啦啦啦


啦啦啦

啦啦啦  %只有空行的行才能起到文字另起一行的作用,且多个空行也只会换一行

······答周公问:
\begin{quote}
\zihao{-3}\kaishu   %像这样修改字号等格式的命令会影响后面整个环境
     勾广三,股修四,径隅五。
\end{quote}         
%类似这样\begin{}
%······    
%\end{}的形式就是一块环境,可以有也可以没有参数
%定理也是一类环境
%打出数学公式也可以用两个$$表示,将公式夹在中间

于是我们得到$\angle A=\pi$

或者也可以用
\begin{equation}
     a+b=c
\end{equation}      %这样用可以在后面有公式号,另外两种则没有
\begin{equation}
2^{10}=1024
\end{equation}
\begin{equation}  
60^\circ            %此标志代表°,也可以用输入法自带的度(°)
\end{equation}

或者这样
\[a^2+b^2=c^2\]  %\[ \]是数学公式用的符号
\begin{quotation}        %quotation会后缩,quote则不会
     勾广三,股修四,径隅五。
\end{quotation}
%使用浮动体插入插图
\begin{figure}[ht]  %参数[ht]表示浮动体可以出现在环境周围的文本所在处(here)和一页的顶部(top)
     \centering
     \includegraphics[width=10cm]{tree.jpg}
     \caption{德国的一种枝干形状很有趣的树}       %caption命令给插图加上自动编号和标题
     \label{fig:tower}        %给图形定义一个标签
\end{figure}

%表格一般都直接在LaTeX中完成的,表格的行、列对齐模式和表格线由tebular环境完成
\begin{tabular}{|lll|}        %|lll|表示左对齐,|rrr|代表右对齐
     \hline
     直角边 $a$ & 直角边 $b$ & 斜边 $c$ \\  %在tabular环境内部,行与行之间用命令\\隔开,每行内部则是由&分开表格中横线用命令\hline产生
     \hline
     3 & 4 & 5 \\
     5 & 12 & 13 \\
     \hline
\end{tabular}

和以上表格对比

\begin{table}[H]         %[H]表示就在此处不再浮动了,此功能在float宏包中
     \begin{tabular}{|rrr|}
          \hline
          直角边 $a$ & 直角边 $b$ & 斜边 $c$ \\  
          \hline
          3 & 4 & 5 \\
          5 & 12 & 13 \\
          \hline 
          \end{tabular}
          \qquad        %此命令产生约2cm的空白
          $(a^2+b^2=c^2)$
%在表格比较小,行文又要求连贯的场合float宏包的这种不浮动的图表环境是很有用的
\end{table}    



%最后是参考文献列表(待续)



\end{document}
%百分号后是注释
%